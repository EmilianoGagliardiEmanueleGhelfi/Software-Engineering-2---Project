%% LyX 2.2.2 created this file.  For more info, see http://www.lyx.org/.
%% Do not edit unless you really know what you are doing.
\documentclass[english]{article}
\usepackage[T1]{fontenc}
\usepackage[latin9]{luainputenc}
\usepackage{float}
\usepackage{graphicx}
\usepackage{babel}
\begin{document}

\section{Architectural design}

\subsection{Overview}
\begin{center}
\begin{figure}[H]
\begin{centering}
\includegraphics[width=0.5\paperwidth]{\string"DD Images/Architectures/Arch Overview\string".png}
\par\end{centering}
\caption{High level architecture}
\end{figure}
\par\end{center}

In this section is presented an high level architecture. It shows
how is structured the system and the main interaction between subsystems.\\
There is a WebServer with Nginx technology that serves static content
(like .html, .js, .css) to browser. When WebServer receives certain
API calls it acts as reverse proxy. Proxying is used to distribute
load among several servers. When web server proxies a request, it
send the request to the Application Server, fetches the response and
sends back to the client. \\
Application server contains the business logic of the software to
be and interacts with the Database Server and with the CarOS.\\
Mobile Application only makes API calls to WebServer and receives
back the JSON response of Application Server.\\
This project starts as a monolithic application because of simplicity
of development, but with particular attention to the modularization.
So, in case will be the need of scaling, the refactoring to micro-services
won't be difficult. 

\subsection{High level components and their interaction}

In this section are presented the high level component and it's described
how they interact with each other.The client component is made of
the Web Browser and the Mobile Application. Both communicates with
the server through its interface.\\
The application server communicates with DBMS through DBMS API and
with CarOS through it's API.\\
In this way the central system always knows the status (position,
battery, etc.) of all cars and also cars can initiate a communication
with server when they need to communicate important event. This is
done with observer pattern as specified later in this document.\\
The communication between client and server can be synchronous or
asynchronous depending on the kind of interaction. The server can
communicate asynchronously with client with notification or messages
(email).This is the class diagram of the data core (likely during
the coding phase a more detailed data structure will come up). There
are two important comsideration to do about this model: the first
one is that the two relations user-reservation and user-ride will
guarantee the reconstruction of users \textquotedblleft story\textquotedblright{}
in the use of the Power Enjoy service, and the second one related
to a bit more technical solution about car system. In the car class
there are some informations that in general shouldn\textquoteright t
be stored, like battery life and position, because they change rapidly.
The explanation of this decision is in the component diagram description,
in particular in the explanation of how the car proxy component works.

\newpage{}

\subsection{High level component and their interaction}

\begin{figure}[H]
\begin{centering}
\includegraphics[width=0.5\paperwidth]{\string"DD Images/Diagrams/High Level component\string".png}
\par\end{centering}
\caption{High level component diagram}
\end{figure}


\subsection{Data model}

\begin{figure}[H]

\begin{centering}
\includegraphics[width=0.5\paperwidth]{\string"DD Images/Diagrams/Model class diagram\string".png}
\par\end{centering}
\caption{Data class diagram}

\end{figure}

This is the core of the data stored in the db. It is not complete,
it contains only the more important information and it is possible
that during the coding phase more structures will come up. There are
two important considerations to do: the first one, is that the two
relations User-Ride and User-Reservation lead to the construction
of the whole story of a user, and the second one, more technical,
is related to the Car class. This class contains some information
that in general aren't stored in a database, because they change rapidly,
like Position and battery life. This decision is motivated and explained
in the description of the car proxy component, in the following section.

\subsection{Component view}
\begin{center}
\begin{figure}[H]
\begin{centering}
\includegraphics[width=0.8\paperwidth]{\string"DD Images/Diagrams/Component diagram (detailed)\string".png}
\par\end{centering}
\caption{More detailed component diagram}
\end{figure}
\par\end{center}

\subsubsection{Car proxy component}

Car proxy is the abstraction of cars in the server. It must be invoked
by other internal subsystems that need information about cars physical
status (real time informations), like position, battery life etc.
This component absolutely doesn't care about information related to
reservations or rides. \\
The main purpose of this component is to request cars using their
API, but its implementations should also guarantee that the systems
working on cars don't have to support high amount of parallel request.
In order to do that, car proxy will use the database, storing a physical
characteristic related to the timestamp of the API call that has provided
that information. When a request arrives, car proxy decides which
data should be provided: the one got through the API call to the car,
or the one stored in the database. If the stored data are sufficiently
recent then they can be provided, otherwise it is needed to make the
request.\\
As it can be seen in the diagram, this component has a considerable
fan-in, it could be necessary in future to make it scalable.

\subsubsection{Car monitoring component}

This component is substantially a daemon. It periodically asks to
the car proxy component the informations that operators need to do
car maintenance. If a damage has been detected, the notification component
should be called, in order to communicate it to an operator.

\subsubsection{Ride manager}

Ride manager is the component that takes care abut the data in the
server corresponding to rides in the real world. Furthermore, when
a ride is set to pause, it use the car proxy component to lock or
unlock the car. 

\subsubsection{Reservation manager}

This component manages users reservation request, setting the status
of a car from available to reserved and vice versa. It should also
take care about reservation expiration, resetting the car state from
reserved to available, and taxing the user through the component payment
manager.

\subsubsection{Registration manager}

This component accepts request from guest of joining the Power Enjoy
service. It checks if the guest has a valid drive license and adds
it to the users. It also verifies the validity of the payment informations
received by the guest using the services provided by the payment manager
component.

\subsubsection{Profile manager}

This component reply to users that want to see their past utilization
of the power enjoy services.

\subsubsection{Customer messages handler}

This component receives messages about users regarding malfunctions
of cars, and use the notification manager to notify the operator that
should take care about that.

\subsubsection{Authentication manager}

The authentication manager the component that manages the login of
user, operator, and administrator, and administrates session.

\subsubsection{Car search engine}

This component interrogates the car proxy component and get cars position,
performing the two type of car research.

\subsubsection{Car manager }

The car manager component can be used by operators to change car status
in the server. For instance, if an operator is going to work on a
car, it should use this component to switch the car status to ``under
maintenance'', and then to unlock it.

\subsubsection{Administrator functionality provider}

This component provides the functionalities accessible only by the
administrators that are listed in the RASD document.

\subsubsection{Payment manager}

This component uses external API of the accepted payment service.
It should be used not only for carry out the payments, but also to
verify during the registration of a user that the provided payment
informations are correct.

\subsubsection{Notification manager}

Notification manager is used to send notifications to users (when
their reserved car changes status), and to operator (when the system
detects that an intervention on a car is needed, or a user make a
communication). 

\subsubsection{Car listener }

This is a subcomponent of the ride component, that is instantiated
for a ride when the car is successfully unlocked. After the unlock
of a car, probably the engine will be turned on. This component fulfil
the problem of getting the engine ignition time, in order to calculate
the correct ride cost. The runtime flow is explained with a sequence
diagram in the section Runtime view. 

\subsection{Deployment view}

The deployment diagram shows the hardware of the system and the software
that it's installed on it.\\
In this diagram there is the Mobile Application installed on Mobile
Phone of the user.\\
The browser runs on user/operator/admin PC.\\
The Application Server and the Web Server are on different Nodes and
on different environment. In this way they are totally decoupled and
it's ensured the scalability of the system. The DBMS runs on a different
node in order to not overload a node with a huge load.\\
If the need of scalability becomes more important for the system there
could be more application server on different nodes with a load balancer
before them. This is compatible with a cloud approach. 
\begin{center}
\begin{figure}[H]
\begin{centering}
\includegraphics[width=0.5\paperwidth]{\string"DD Images/Diagrams/Deployment Diagram\string".png}
\par\end{centering}
\caption{Deployment view}

\end{figure}
\par\end{center}

\subsection{Runtime view}

\subsubsection{Registration}

\begin{figure}[H]
\begin{centering}
\includegraphics[width=0.7\paperwidth]{\string"DD Images/Diagrams/Runtime - Registration\string".png}
\par\end{centering}
\caption{Registration runtime view}
\end{figure}

The client application request is sent to the registration manager
component, that have to verify that the payment information and the
driver license are valid. To do this, as mentioned in the product
perspective section of the RASD document, external services are used.
Then, if the information are valid, a new user is created and inserted
in the database, and the result of the operation is sent to the user.

\subsubsection{Car search}

\begin{figure}[H]

\begin{centering}
\includegraphics[width=0.7\paperwidth]{\string"DD Images/Diagrams/Runtime - Car Search\string".png}
\par\end{centering}
\caption{Car search runtime view}
\end{figure}

The user through the application insert the information needed in
order to do the research, then the information are sent to the search
engine that performs the research using the algorithm described in
the algorithm section of this document. The position of the cars are
obtained interrogating the car proxy component.

\subsubsection{Reservation}

\begin{figure}[H]
\begin{centering}
\includegraphics[width=0.7\paperwidth]{\string"DD Images/Diagrams/Runtime - Reservation\string".png}
\par\end{centering}
\caption{Car reservation runtime view}
\end{figure}

When the request is received the reservation manager check if it the
user can reserve the car identified by carID, and send back the outcome.
Then if the reservation is allowed, the reservation manager delegates
to the ride manager the managing of the car picking up.

\subsubsection{Unlock car}

\begin{figure}[H]

\begin{centering}
\includegraphics[width=0.7\paperwidth]{\string"DD Images/Diagrams/Runtime - Unlock Car\string".png}
\par\end{centering}
\caption{Unlock car runtime view}
\end{figure}

When a client make the unlock request, the system verifies that the
request owner is allowed to unlock the specified car. This operation
is carried out checking two things: that the user has actually reserved
the car, and that it is sufficiently close to that car. Then through
the car proxy the car is unlocked, and the outcome is sent to the
user.

\subsubsection{End ride}

\begin{figure}[H]

\begin{centering}
\includegraphics[width=0.7\paperwidth]{\string"DD Images/Diagrams/Runtime - EndRide\string".png}
\par\end{centering}
\caption{End ride runtime view}
\end{figure}

The stopEngineCallback remote method is called by the car system on
the subcomponent of the ride manager (car listener) when the stop
of the engine is detected. The call contains as parameter the cadID,
so the ride manager component can find in the persistent data the
user that is riding the car. In this way the system can ask to a user
that has stop the engine if it wants to end the ride, or put it in
pause (it is necessary to manage the situations in which there is
no reply). Then if the ride is finished, the payment can be carried
out through the payment method API call.

\subsubsection{Car monitoring}

\begin{figure}[H]

\begin{centering}
\includegraphics[width=0.7\paperwidth]{\string"DD Images/Diagrams/Runtime - Car Monitoring\string".png}
\par\end{centering}
\caption{Car monitoring runtime view}
\end{figure}

Here is explained how the system try to keep cars in a good state.
The car monitoring component makes periodic requests to the cars system,
asking if the system needs some kind of operators intervention. If
for instance a malfunction is detected, or the battery level is too
low, the central system can notify the operator that is working in
that area.

\subsubsection{Communication of a malfunction}

\begin{figure}[H]

\begin{centering}
\includegraphics[width=0.7\paperwidth]{\string"DD Images/Diagrams/Runtime - Communicate Malfunction\string".png}
\par\end{centering}
\caption{Communication of a malfunction runtime view}

\end{figure}

In this diagram is explained how a user can communicate to the system
eventual cars problem. Further, when an operator decides to work on
a car, it can set its status to ``under maintenance'', so no user
can reserve it.

\subsection{Component interface}

\subsection{Selected architectural styles and patterns}

\subsubsection{Tiers}

The system will be divided into 5 tiers:
\begin{enumerate}
\item Database Server 
\item Application Server
\item Web Server
\item Mobile Application
\item Car
\end{enumerate}
Note that here the car is considered as a tier but actually is a blackbox
capable of receive request from our system and to call some interfaces
provided by components of the system. \\
This division is in order to not overload any machine. Application
Server and Web Server are deployed on different machine because if
due to great load there will be the need of another Application Server
will be easier to instantiate another machine.

\subsubsection{Layers}

As said before the system will be divided into 3 Layers:
\begin{itemize}
\item Presentation Layer: it's distributed on the mobile application or
on the Web Server.
\item Business Layer: it's on the Application Server.
\item Data Layer: it's on the Database Server.
\end{itemize}
Divide layers is important for decoupling and for dividing responsibilities.
In this way different developers can focus on different tasks abstracting
from other layers. Diving in layers is also important for mental clearness
because in the case of a problem it should be clear where the problem
is located.

\subsubsection{Protocols}

In this section is described how different tier communicate with each
other and how they exchange data.

\subsubsection*{JDBC}

Used by the Application Server to communicate with the database server.
JDBC API is the standard for database-independent connectivity between
the Java programming language and a wide range of databases SQL databases
and other tabular data sources. The JDBC API provides a call-level
API for SQL-based database access.

\subsubsection*{RESTful API}

Used by both Mobile Application and Web Application to access the
services provided by Application Server component.

\subsubsection{Design Patterns}

\subsubsection*{Client-Server}

Client server is the base of the architecture. The central system
is the server and provides services to both web application and mobile
application. The central server communicates with car that acts as
a server with respect to central system. 

\subsubsection*{Monolithic}

It's decided to use a monolithic approach because of simplicity of
development. Particular attention is give to decoupling and modularization.
In this way the refactoring to micro-services will be easy if there
will be the need of scalability. With the attention to decoupling
a cloud approach it's possible and it's possible to deploy different
part of the system on different machines.

\subsubsection*{Proxy}

Proxy Pattern is used to communicate with the car. Car Proxy is the
abstraction of the car in the system. When a component needs to communicate
with the car, it calls the method provided by car proxy. Then car
proxy decides wether to call car API or to get information from the
DB.

\subsubsection*{Observer}

Observer Pattern is used to get callback from car's important event
like engine start or engine stop. CarObserver is instantiated when
the user unlocks the car and the car is informed that it needs to
call the observer when an important event takes place.
\end{document}
